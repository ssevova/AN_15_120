\section{Control Region Closure Studies}
\label{app:CR}

In this section, we demonstrate the validity of the extracted $\met$ distributions for the background control regions as defined in Sec.~\ref{subsec:bkg_hadronic} and Sec.~\ref{subsec:bkg_semilept}. By comparing the $\met$ distribution in MC for the various background processes in the defined signal and control regions, we validate the $\met$ shape templates employed in the analysis.

\subsection{Control region for \texorpdfstring{$\ttbar$}{ttbar}}
\label{app:CRttbar}

As demonstrated in Sec.~\ref{subsubsec:bkg_hadronic_ttbar} semileptonic \ttbar\: is the largest background in the hadronic categories. We determine that the mis-identification of the required ``Tight" lepton is due its \pt falling below the $30\:\GeV\:$ threshold, or that it fails ID requirements. We check the lepton $\eta$ for semileptonic $\ttbar$ in the signal region after removing the lepton veto cut and show that the lepton falls within detector acceptance $98\%$ of the time.

\iffalse
\begin{figure}
\centering
\includegraphics[textwidth=0.48\textwidth]{figure: }

\caption{}


\end{figure}
\fi

The closure tests performed in the electron and muon channel in semileptonic \ttbar\: shown in Fig:~\ref{fig:}, demonstrate good agreement between the $\met$ distributions in the signal and control regions in both channels. 

\subsubsection{Control region for \texorpdfstring{$V+$jets}{Vjets}}
\label{app:CRZjets}

As demonstrated in Fig.~\ref{}, the agreement between the $\met$ distribution in the signal region and the $\met$ distribution in the non-lepton subtracted $V+$jets control region demonstrates that we are able to use the $\met$ template without removing the lepton four-momentum in the control region.


\iffalse
\begin{figure}
\centering
\includegraphics[textwidth=0.48\textwidth]{figure: }

\caption{}


\end{figure}
\fi
